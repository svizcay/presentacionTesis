\begin{frame}{Thread}
\begin{block}{Principales características}
\begin{itemize}
  \item Comienza su ejecución desde el momento en que se declara una variable del tipo \textit{thread}.
  \item La forma más sencilla de indicar el código a ejecutar, es pasándole al constructor el nombre de la función deseada.
  \item Otras alternativas son especificar un método de un objeto, un método estático de una clase, 
  \item Se \textbf{DEBE SIEMPRE} decidir si se desea esperar a que la nueva hebra finalice su trabajo o si es que se desea permitir que corra libremente.
\end{itemize}
\end{block}
\end{frame}
%==================================================================================

\begin{frame}{Thread}
\begin{block}{Join o detach}
\begin{itemize}
  \item Si se desea esperar a que la nueva hebra finalice su trabajo antes de llamar al destructor, llamaremos al método \textbf{join} del objeto \textit{thread} recien creado.
  \item En caso en que se decide dejar a la nueva hebra correr de forma completamente independiente, perrmitiéndonos incluso a que la hebra principal finalice su trabajo y \textit{libere} sus variables locales, llamaremos al método \textbf{detach}.
\end{itemize}
\begin{itemize}
  \item En el caso en que se llame al método \textit{deatch}, la nueva hebra de ejecución \textbf{se desliga} del objeto thread que recién creamos y se continua ejecutando en background como un \textit{daemon}.
\end{itemize}
\end{block}
\end{frame}
%==================================================================================

\begin{frame}{Thread}
\begin{block}{Más sobre detach}
\begin{itemize}
  \item Si se desliga la hebra de ejecución del objeto \textit{thread} con \textit{detach} y la hebra principal ejecutando \textit{main} finaliza, no se puede asegurar si la nueva hebra de ejecución continuará ejecutándose.
  \item Caso específico: salidas a \textit{cout} o a ficheros no son realizadas por la segunda hebra si es que la primera finaliza la ejecución de \textit{main}. Tampoco es reportada su ejecución con el comando \textit{ps H} (comando que muestra la lista de procesos/hebras ejecutándose).
  \item Acorde a \textit{cppreference.com}, ejecución de la segunda hebra debería continuar independientemente.
\end{itemize}
\end{block}
\end{frame}
%==================================================================================

% si el frame contiene un listing, el frame debe ser "fragile"
\begin{frame}[fragile]{Thread}
\begin{block}{Ejemplo básico}
\begin{itemize}
  \item \#include \textless thread\textgreater
  \item Nombre de una función.
  \item Uso de join.
\end{itemize}
\begin{lstlisting}[language=C++, basicstyle=\small]
#include <thread>
void foo();
int main(int argc, char *argv[]) {
  std::thread myThread (foo);
  myThread.join();
  return 0;
}
\end{lstlisting}
\end{block}
\end{frame}
%==================================================================================

% si el frame contiene un listing, el frame debe ser "fragile"
\begin{frame}[fragile]{Thread}
\begin{block}{Segundo ejemplo: función con parámetros}
\begin{itemize}
  \item \textbf{NOTA:} se envían \textbf{copias} de los argumentos aun cuando el prototipo de la función declare que recibe \textbf{referencias} como parámetros.
\end{itemize}
\begin{lstlisting}[language=C++, basicstyle=\small]
#include <thread>
void foo(int a, int b);
int main(int argc, char *argv[]) {
  int x = 5, y = 10;
  std::thread myThread (foo, x, y);
  myThread.join();
  return 0;
}
\end{lstlisting}
\end{block}
\end{frame}

%==================================================================================

% si el frame contiene un listing, el frame debe ser "fragile"
\begin{frame}[fragile]{Thread}
\begin{block}{Tercer ejemplo: método de un objeto}
\begin{itemize}
  \item Recordar que un método no es más que una función que tiene como \textbf{primer parámetro implícito} un puntero constante al objeto llamado \textbf{this}.
\end{itemize}
\begin{lstlisting}[language=C++, basicstyle=\small]
class MyClass {
public:
  MyClass();
  ~MyClass();
  method1(int x);
};
\end{lstlisting}
\end{block}
\end{frame}

% si el frame contiene un listing, el frame debe ser "fragile"
\begin{frame}[fragile]{Thread}
\begin{block}{Tercer ejemplo: método de un objeto}
\begin{itemize}
  \item Recordar que un método no es más que una función que tiene como \textbf{primer parámetro implícito} un puntero constante al objeto llamado \textbf{this}.
\end{itemize}
\begin{lstlisting}[language=C++, basicstyle=\small]
#include <thread>
#include "myclass.hpp"
int main(int argc, char *argv[]) {
  MyClass myObject ();
  int number = 5;
  std::thread myThread (&MyClass::method1, &myObject, number);
  myThread.join();
  return 0;
}
\end{lstlisting}
\end{block}
\end{frame}
%==================================================================================

% si el frame contiene un listing, el frame debe ser "fragile"
\begin{frame}[fragile]{Thread}
\begin{block}{Cuarto ejemplo: función con parámetros de tipo referencias}
\begin{itemize}
  \item \#include \textless functional\textgreater  para usar las funciones ref y cref.
  \item Usar \textbf{ref} para enviar la variable por referencia y \textbf{cref} para cuando la referencia fue declarada como \textbf{const}.
\end{itemize}
\begin{lstlisting}[language=C++, basicstyle=\small]
#include <thread>
#include <functional>
void foo(int &a, const int &b);
int main(int argc, char *argv[]) {
  int x = 5, y = 10;
  std::thread myThread (foo, std::ref(x), std::cref(y));
  myThread.join();
  return 0;
}
\end{lstlisting}
\end{block}
\end{frame}
%==================================================================================

% si el frame contiene un listing, el frame debe ser "fragile"
\begin{frame}{Thread}
\begin{block}{Quinto ejemplo: crear un objeto con hebra de ejecución propia}
\begin{itemize}
  \item Se debe declarar un \textbf{atributo miembro} del tipo \textbf{thread} dentro de la clase.
  \item Se llama al constructor de \textit{thread}, indicando el método a ejecutar utilizando la nueva hebra de ejecución, al momento en que se \textbf{construye} un objeto de la clase.
  \item Se invoca al método \textbf{join} del objeto \textit{thread} dentro del \textbf{destructor} de la clase, el cual será ejecutado una vez que se acabe el \textit{scope} del objeto concurrente o a través de una llamada explícita a \textit{delete}.
\end{itemize}
\end{block}
\end{frame}

% si el frame contiene un listing, el frame debe ser "fragile"
\begin{frame}[fragile]{Thread}
\begin{block}{Quinto ejemplo: crear un objeto con hebra de ejecución propia}
\begin{lstlisting}[language=C++, basicstyle=\small]
#include <thread>

class Task {
public:
  Task();
  ~Task();

private:
  std::thread privateThread;
  void main();
};
\end{lstlisting}
\end{block}
\end{frame}

% si el frame contiene un listing, el frame debe ser "fragile"
\begin{frame}[fragile]{Thread}
\begin{block}{Quinto ejemplo: crear un objeto con hebra de ejecución propia}
\begin{lstlisting}[language=C++, basicstyle=\small]
#include "task.hpp"

Task::Task() {
  privateThread = std::thread(&Task::main, this);
}

Task::~Task() {
  privateThread.join();
}
\end{lstlisting}
\end{block}
\end{frame}
%==================================================================================
\begin{frame}{Thread}
\begin{block}{Qué es lo que se envía realmente al constructor de la clase thread?}
\begin{itemize}
  \item Cuando se crea un objeto thread y se indica una función en conjunto con los argumentos que necesita, lo que se envía realmente al constructor es la función retornada por la utilidad \textbf{bind}.
  \item Bind es una utilidad que nos permite generar funciones a partir de otra función, la cual es utilizada como plantilla.
  \item Bind puede ligar valores a los parámetros que son esperados por la función plantilla.
\end{itemize}
\end{block}
\end{frame}

\begin{frame}[fragile]{Thread}
\begin{block}{Usando bind para generar funciones}
  \begin{columns}
  \begin{column}{.45\textwidth}
\begin{lstlisting}[language=C++, basicstyle=\small]
int sum(int a, int b) {
  return a + b;
}

int sum34() {
  return 3 + 4;
}
\end{lstlisting}
  \end{column}
  \begin{column}{.45\textwidth}
\begin{lstlisting}[language=C++, basicstyle=\small]
int sumA4(int a) {
  return a + 4;
}

int sum3B(int b) {
  return 3 + b;
}
\end{lstlisting}
  \end{column}
  \end{columns}
\end{block}
\end{frame}

\begin{frame}[fragile]{Thread}
\begin{block}{Usando bind para generar funciones}
\begin{lstlisting}[language=C++, basicstyle=\small]
#include <functional>
int main(int argc, char *argv[]) {
  auto bindSum34 = std::bind(sum, 3, 4);
  auto bindSumA4 = std::bind(sum, std::placeholders::_1, 4);
  auto bindSum3B = std::bind(sum, 3, std::placeholders::_1);
  std::cout << sum(3, 4) << std::endl;
  std::cout << bindSum34() << std::endl;
  std::cout << bindSumA4(3) << std::endl;
  std::cout << bindSum3B(4) << std::endl;

  return 0;
}
\end{lstlisting}
\end{block}
\end{frame}

\begin{frame}{Thread}
\begin{block}{Comentarios adicionales sobre bind}
\begin{itemize}
  \item Tanto \textbf{bind} como \textbf{placeholders} están declarados en el header \textbf{functional}.
  \item Se pueden especificar placeholders adicionales. \_1 hace referencia al primer parámetro, \_2 al segundo y así sucesivamente.
  \item Las variables que reciben la función retornada por bind funcionan exactamente igual que las funciones ordinarias, y deben invocarse especificando una lista de argumentos encerrada entre paréntesis.
\end{itemize}
\end{block}
\end{frame}

\begin{frame}{Thread}
\begin{block}{Comentarios adicionales sobre bind}
\begin{itemize}
  \item No necesitamos explicitar el tipo de valor retornado por bind en el momento de declarar las variables. Una de las nuevas características de C++11 es el uso de \textbf{auto} para dejar al compilador que realice estas detecciones.
  \item El uso de bind es útil para modificar el \textit{signature} de funciones a las cuales no tenemos acceso para modificarlas, como por ejemplo algoritmos provistos por la stl en el header algorithms.
\end{itemize}
\end{block}
\end{frame}
%==================================================================================

\begin{frame}{Thread}
\begin{block}{Identificando a una hebra}
\begin{itemize}
  \item Además de los métodos \textit{join} y \textit{detach}, los objetos del tipo \textit{thread} poseen un tercer método que resulta de interés, el método \textbf{get\_id}.
  \item El método get\_id retorno un identificador único de la hebra de ejecución.
  \item ID retornado resulta ser una \textbf{secuencia numérica larga} poco amigable para ser recordada.
  \item El ID retornado por get\_id puede ser \textbf{reutilizado} a medida en que se van creando y destruyendo las hebras de ejecución.
  \item Se aconseja asignar manualmente IDs únicos a cada thread, agregando a la lista de parámetros una variable numérica que sea utilizada como identificador.
\end{itemize}
\end{block}
\end{frame}

\begin{frame}[fragile]{Thread}
\begin{block}{Identificando a una hebra}
\begin{itemize}
  \item La \textbf{función get\_id} del \textbf{scope this\_thread} no es más que una función que invoca al \textbf{método get\_id} del objeto thread actualmente ejecutando la línea particular de código.
\end{itemize}
\begin{lstlisting}[language=C++, basicstyle=\small]
#include <thread>
void foo(int tid);
int main(int argc, char *argv[]) {
  int id = 0;
  std::thread myThread (foo, id);
  myThread.join();

  return 0;
}
\end{lstlisting}
\end{block}
\end{frame}

\begin{frame}[fragile]{Thread}
\begin{block}{Identificando a una hebra}
\begin{lstlisting}[language=C++, basicstyle=\small]
void foo(int tid) {
  std::cout << "my id: " << std::this_thread::get_id();
  std::cout << std::endl;
  std::cout << "manual id: " << tid << std::endl;
}
\end{lstlisting}
\begin{lstlisting}[language=C++, caption={output}, basicstyle=\small]
my id: 140150155630464
manual id: 0
\end{lstlisting}
\end{block}
\end{frame}
%==================================================================================

\begin{frame}{Thread}
\begin{block}{Otras utilidades del namespace this\_thread}
\begin{itemize}
  \item El namespace \textbf{this\_thread} agrupa un conjunto de funciones básicas que resultan ser útiles para mecanismos de planificación de hebras.
  \item Además de la función \textbf{get\_id}, se encuentran las funciones \textbf{sleep\_for}, \textbf{sleep\_until} y \textbf{yield}.
  \item Todas estas funciones hacen referencia a la hebra actual que se encuentra ejecutándose.
\end{itemize}
\end{block}
\end{frame}

\begin{frame}{Thread}
\begin{block}{Otras utilidades del namespace this\_thread}
\begin{itemize}
  \item La función \textbf{yield} sirve para indicar a la implementación que buscamos que se realice una \textbf{replanificación del scheduling de hebras}, es decir, se da la oportunidad a que se ejecuten otras hebras.
  \item No se puede especificar hebras en particular (la función no recibe ningún parámetro).
  \item No se asegura que otra hebra entre a ejecutarse. Yield debe verse como una \textbf{sugerencia} de replanificación.
\end{itemize}
\end{block}
\end{frame}

\begin{frame}{Thread}
\begin{block}{Otras utilidades del namespace this\_thread}
\begin{itemize}
  \item Uso de sleep y usleep para suspender la ejecución de una hebra es considerado obsoleto.
  \item Uso del namespace \textbf{chrono} para definir periodos de tiempo de forma flexible (precisión de segundos,minutos, horas, microsegundos, nanosegundos, etc.).
  \item Funciones \textbf{sleep\_for} y \textbf{sleep\_until} toman como argumento un \textbf{periodo de tiempo} y un \textbf{instante en el tiempo} respectivamente. El primero es un tiempo \textbf{relativo} al tiempo exacto en que se invoca a sleep\_for y el segundo es un tiempo \textbf{absoluto}, independiente del tiempo en que se invocó a sleep\_until.
\end{itemize}
\end{block}
\end{frame}

\begin{frame}[fragile]{Thread}
\begin{block}{Ejemplo de uso de sleep\_for}
\begin{lstlisting}[language=C++, basicstyle=\small]
#include <chrono>
#include <thread>

int main(int argc, char *argv[]) {
  std::chrono::milliseconds duration (5000); // 5 seconds
  std::this_thread::sleep_for(duration);
  
  return 0;
}
\end{lstlisting}
\end{block}
\end{frame}

\begin{frame}{Thread}
\begin{block}{Consideración con respecto a locks y similares}
\begin{itemize}
  \item Si se está trabajando con \textbf{variables de condición} o \textbf{locks}, el invocar a yield o a a algún tipo de sleep \textbf{no liberará el lock} del mutex que se había adquirido, por lo que se debe tener cuidado a la hora de llamar a estas funciones habiendo adquirido algún lock previamente.
\end{itemize}
\end{block}
\end{frame}