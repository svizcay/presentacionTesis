\documentclass[xcolor=table]{beamer}
\usepackage[utf8]{inputenc}
\usepackage{listings}
\usepackage{xcolor} % for setting colors
\usepackage{booktabs}	% for tables toprule, midrule and bottomrule
%\usepackage[]{algorithm2e}
%\usepackage{algpseudocode}
%\usepackage{algorithm}
\usetheme{ZBH}

% set the default code style
\lstset{
	frame=tb, % draw a frame at the top and bottom of the code block
    tabsize=2, % tab space width
    showstringspaces=false, % don't mark spaces in strings
    numbers=left, % display line numbers on the left
    commentstyle=\color{green}, % comment color
    keywordstyle=\color{blue}, % keyword color
    stringstyle=\color{orange} % string color
}

\author{Sebastián Vizcay}
\title{Simulación de Monte Carlo paralela para sistemas ferromagnéticos en un modelo de Heisenberg incluyendo interacciones dipolares}
\date{\today}

\begin{document}

\begin{frame}

\titlepage
\end{frame}

\begin{frame}{Paralelización de simulación de Monte Carlo}
\begin{block}{Tabla de contenidos}
\begin{itemize}
  \item Introducción.
  \item Marco teórico.
  \item Arquitecturas y modelos de computación paralela.
  \item Trabajo realizado.
  \item Experimentos y resultados.
  \item Conclusiones.
\end{itemize}
\end{block}
%	\tableofcontents
\end{frame}

%\section{el header thread}
%\begin{frame}{Thread}
\begin{block}{Principales características}
\begin{itemize}
  \item Comienza su ejecución desde el momento en que se declara una variable del tipo \textit{thread}.
  \item La forma más sencilla de indicar el código a ejecutar, es pasándole al constructor el nombre de la función deseada.
  \item Otras alternativas son especificar un método de un objeto, un método estático de una clase, 
  \item Se \textbf{DEBE SIEMPRE} decidir si se desea esperar a que la nueva hebra finalice su trabajo o si es que se desea permitir que corra libremente.
\end{itemize}
\end{block}
\end{frame}
%==================================================================================

\begin{frame}{Thread}
\begin{block}{Join o detach}
\begin{itemize}
  \item Si se desea esperar a que la nueva hebra finalice su trabajo antes de llamar al destructor, llamaremos al método \textbf{join} del objeto \textit{thread} recien creado.
  \item En caso en que se decide dejar a la nueva hebra correr de forma completamente independiente, perrmitiéndonos incluso a que la hebra principal finalice su trabajo y \textit{libere} sus variables locales, llamaremos al método \textbf{detach}.
\end{itemize}
\begin{itemize}
  \item En el caso en que se llame al método \textit{deatch}, la nueva hebra de ejecución \textbf{se desliga} del objeto thread que recién creamos y se continua ejecutando en background como un \textit{daemon}.
\end{itemize}
\end{block}
\end{frame}
%==================================================================================

\begin{frame}{Thread}
\begin{block}{Más sobre detach}
\begin{itemize}
  \item Si se desliga la hebra de ejecución del objeto \textit{thread} con \textit{detach} y la hebra principal ejecutando \textit{main} finaliza, no se puede asegurar si la nueva hebra de ejecución continuará ejecutándose.
  \item Caso específico: salidas a \textit{cout} o a ficheros no son realizadas por la segunda hebra si es que la primera finaliza la ejecución de \textit{main}. Tampoco es reportada su ejecución con el comando \textit{ps H} (comando que muestra la lista de procesos/hebras ejecutándose).
  \item Acorde a \textit{cppreference.com}, ejecución de la segunda hebra debería continuar independientemente.
\end{itemize}
\end{block}
\end{frame}
%==================================================================================

% si el frame contiene un listing, el frame debe ser "fragile"
\begin{frame}[fragile]{Thread}
\begin{block}{Ejemplo básico}
\begin{itemize}
  \item \#include \textless thread\textgreater
  \item Nombre de una función.
  \item Uso de join.
\end{itemize}
\begin{lstlisting}[language=C++, basicstyle=\small]
#include <thread>
void foo();
int main(int argc, char *argv[]) {
  std::thread myThread (foo);
  myThread.join();
  return 0;
}
\end{lstlisting}
\end{block}
\end{frame}
%==================================================================================

% si el frame contiene un listing, el frame debe ser "fragile"
\begin{frame}[fragile]{Thread}
\begin{block}{Segundo ejemplo: función con parámetros}
\begin{itemize}
  \item \textbf{NOTA:} se envían \textbf{copias} de los argumentos aun cuando el prototipo de la función declare que recibe \textbf{referencias} como parámetros.
\end{itemize}
\begin{lstlisting}[language=C++, basicstyle=\small]
#include <thread>
void foo(int a, int b);
int main(int argc, char *argv[]) {
  int x = 5, y = 10;
  std::thread myThread (foo, x, y);
  myThread.join();
  return 0;
}
\end{lstlisting}
\end{block}
\end{frame}

%==================================================================================

% si el frame contiene un listing, el frame debe ser "fragile"
\begin{frame}[fragile]{Thread}
\begin{block}{Tercer ejemplo: método de un objeto}
\begin{itemize}
  \item Recordar que un método no es más que una función que tiene como \textbf{primer parámetro implícito} un puntero constante al objeto llamado \textbf{this}.
\end{itemize}
\begin{lstlisting}[language=C++, basicstyle=\small]
class MyClass {
public:
  MyClass();
  ~MyClass();
  method1(int x);
};
\end{lstlisting}
\end{block}
\end{frame}

% si el frame contiene un listing, el frame debe ser "fragile"
\begin{frame}[fragile]{Thread}
\begin{block}{Tercer ejemplo: método de un objeto}
\begin{itemize}
  \item Recordar que un método no es más que una función que tiene como \textbf{primer parámetro implícito} un puntero constante al objeto llamado \textbf{this}.
\end{itemize}
\begin{lstlisting}[language=C++, basicstyle=\small]
#include <thread>
#include "myclass.hpp"
int main(int argc, char *argv[]) {
  MyClass myObject ();
  int number = 5;
  std::thread myThread (&MyClass::method1, &myObject, number);
  myThread.join();
  return 0;
}
\end{lstlisting}
\end{block}
\end{frame}
%==================================================================================

% si el frame contiene un listing, el frame debe ser "fragile"
\begin{frame}[fragile]{Thread}
\begin{block}{Cuarto ejemplo: función con parámetros de tipo referencias}
\begin{itemize}
  \item \#include \textless functional\textgreater  para usar las funciones ref y cref.
  \item Usar \textbf{ref} para enviar la variable por referencia y \textbf{cref} para cuando la referencia fue declarada como \textbf{const}.
\end{itemize}
\begin{lstlisting}[language=C++, basicstyle=\small]
#include <thread>
#include <functional>
void foo(int &a, const int &b);
int main(int argc, char *argv[]) {
  int x = 5, y = 10;
  std::thread myThread (foo, std::ref(x), std::cref(y));
  myThread.join();
  return 0;
}
\end{lstlisting}
\end{block}
\end{frame}
%==================================================================================

% si el frame contiene un listing, el frame debe ser "fragile"
\begin{frame}{Thread}
\begin{block}{Quinto ejemplo: crear un objeto con hebra de ejecución propia}
\begin{itemize}
  \item Se debe declarar un \textbf{atributo miembro} del tipo \textbf{thread} dentro de la clase.
  \item Se llama al constructor de \textit{thread}, indicando el método a ejecutar utilizando la nueva hebra de ejecución, al momento en que se \textbf{construye} un objeto de la clase.
  \item Se invoca al método \textbf{join} del objeto \textit{thread} dentro del \textbf{destructor} de la clase, el cual será ejecutado una vez que se acabe el \textit{scope} del objeto concurrente o a través de una llamada explícita a \textit{delete}.
\end{itemize}
\end{block}
\end{frame}

% si el frame contiene un listing, el frame debe ser "fragile"
\begin{frame}[fragile]{Thread}
\begin{block}{Quinto ejemplo: crear un objeto con hebra de ejecución propia}
\begin{lstlisting}[language=C++, basicstyle=\small]
#include <thread>

class Task {
public:
  Task();
  ~Task();

private:
  std::thread privateThread;
  void main();
};
\end{lstlisting}
\end{block}
\end{frame}

% si el frame contiene un listing, el frame debe ser "fragile"
\begin{frame}[fragile]{Thread}
\begin{block}{Quinto ejemplo: crear un objeto con hebra de ejecución propia}
\begin{lstlisting}[language=C++, basicstyle=\small]
#include "task.hpp"

Task::Task() {
  privateThread = std::thread(&Task::main, this);
}

Task::~Task() {
  privateThread.join();
}
\end{lstlisting}
\end{block}
\end{frame}
%==================================================================================
\begin{frame}{Thread}
\begin{block}{Qué es lo que se envía realmente al constructor de la clase thread?}
\begin{itemize}
  \item Cuando se crea un objeto thread y se indica una función en conjunto con los argumentos que necesita, lo que se envía realmente al constructor es la función retornada por la utilidad \textbf{bind}.
  \item Bind es una utilidad que nos permite generar funciones a partir de otra función, la cual es utilizada como plantilla.
  \item Bind puede ligar valores a los parámetros que son esperados por la función plantilla.
\end{itemize}
\end{block}
\end{frame}

\begin{frame}[fragile]{Thread}
\begin{block}{Usando bind para generar funciones}
  \begin{columns}
  \begin{column}{.45\textwidth}
\begin{lstlisting}[language=C++, basicstyle=\small]
int sum(int a, int b) {
  return a + b;
}

int sum34() {
  return 3 + 4;
}
\end{lstlisting}
  \end{column}
  \begin{column}{.45\textwidth}
\begin{lstlisting}[language=C++, basicstyle=\small]
int sumA4(int a) {
  return a + 4;
}

int sum3B(int b) {
  return 3 + b;
}
\end{lstlisting}
  \end{column}
  \end{columns}
\end{block}
\end{frame}

\begin{frame}[fragile]{Thread}
\begin{block}{Usando bind para generar funciones}
\begin{lstlisting}[language=C++, basicstyle=\small]
#include <functional>
int main(int argc, char *argv[]) {
  auto bindSum34 = std::bind(sum, 3, 4);
  auto bindSumA4 = std::bind(sum, std::placeholders::_1, 4);
  auto bindSum3B = std::bind(sum, 3, std::placeholders::_1);
  std::cout << sum(3, 4) << std::endl;
  std::cout << bindSum34() << std::endl;
  std::cout << bindSumA4(3) << std::endl;
  std::cout << bindSum3B(4) << std::endl;

  return 0;
}
\end{lstlisting}
\end{block}
\end{frame}

\begin{frame}{Thread}
\begin{block}{Comentarios adicionales sobre bind}
\begin{itemize}
  \item Tanto \textbf{bind} como \textbf{placeholders} están declarados en el header \textbf{functional}.
  \item Se pueden especificar placeholders adicionales. \_1 hace referencia al primer parámetro, \_2 al segundo y así sucesivamente.
  \item Las variables que reciben la función retornada por bind funcionan exactamente igual que las funciones ordinarias, y deben invocarse especificando una lista de argumentos encerrada entre paréntesis.
\end{itemize}
\end{block}
\end{frame}

\begin{frame}{Thread}
\begin{block}{Comentarios adicionales sobre bind}
\begin{itemize}
  \item No necesitamos explicitar el tipo de valor retornado por bind en el momento de declarar las variables. Una de las nuevas características de C++11 es el uso de \textbf{auto} para dejar al compilador que realice estas detecciones.
  \item El uso de bind es útil para modificar el \textit{signature} de funciones a las cuales no tenemos acceso para modificarlas, como por ejemplo algoritmos provistos por la stl en el header algorithms.
\end{itemize}
\end{block}
\end{frame}
%==================================================================================

\begin{frame}{Thread}
\begin{block}{Identificando a una hebra}
\begin{itemize}
  \item Además de los métodos \textit{join} y \textit{detach}, los objetos del tipo \textit{thread} poseen un tercer método que resulta de interés, el método \textbf{get\_id}.
  \item El método get\_id retorno un identificador único de la hebra de ejecución.
  \item ID retornado resulta ser una \textbf{secuencia numérica larga} poco amigable para ser recordada.
  \item El ID retornado por get\_id puede ser \textbf{reutilizado} a medida en que se van creando y destruyendo las hebras de ejecución.
  \item Se aconseja asignar manualmente IDs únicos a cada thread, agregando a la lista de parámetros una variable numérica que sea utilizada como identificador.
\end{itemize}
\end{block}
\end{frame}

\begin{frame}[fragile]{Thread}
\begin{block}{Identificando a una hebra}
\begin{itemize}
  \item La \textbf{función get\_id} del \textbf{scope this\_thread} no es más que una función que invoca al \textbf{método get\_id} del objeto thread actualmente ejecutando la línea particular de código.
\end{itemize}
\begin{lstlisting}[language=C++, basicstyle=\small]
#include <thread>
void foo(int tid);
int main(int argc, char *argv[]) {
  int id = 0;
  std::thread myThread (foo, id);
  myThread.join();

  return 0;
}
\end{lstlisting}
\end{block}
\end{frame}

\begin{frame}[fragile]{Thread}
\begin{block}{Identificando a una hebra}
\begin{lstlisting}[language=C++, basicstyle=\small]
void foo(int tid) {
  std::cout << "my id: " << std::this_thread::get_id();
  std::cout << std::endl;
  std::cout << "manual id: " << tid << std::endl;
}
\end{lstlisting}
\begin{lstlisting}[language=C++, caption={output}, basicstyle=\small]
my id: 140150155630464
manual id: 0
\end{lstlisting}
\end{block}
\end{frame}
%==================================================================================

\begin{frame}{Thread}
\begin{block}{Otras utilidades del namespace this\_thread}
\begin{itemize}
  \item El namespace \textbf{this\_thread} agrupa un conjunto de funciones básicas que resultan ser útiles para mecanismos de planificación de hebras.
  \item Además de la función \textbf{get\_id}, se encuentran las funciones \textbf{sleep\_for}, \textbf{sleep\_until} y \textbf{yield}.
  \item Todas estas funciones hacen referencia a la hebra actual que se encuentra ejecutándose.
\end{itemize}
\end{block}
\end{frame}

\begin{frame}{Thread}
\begin{block}{Otras utilidades del namespace this\_thread}
\begin{itemize}
  \item La función \textbf{yield} sirve para indicar a la implementación que buscamos que se realice una \textbf{replanificación del scheduling de hebras}, es decir, se da la oportunidad a que se ejecuten otras hebras.
  \item No se puede especificar hebras en particular (la función no recibe ningún parámetro).
  \item No se asegura que otra hebra entre a ejecutarse. Yield debe verse como una \textbf{sugerencia} de replanificación.
\end{itemize}
\end{block}
\end{frame}

\begin{frame}{Thread}
\begin{block}{Otras utilidades del namespace this\_thread}
\begin{itemize}
  \item Uso de sleep y usleep para suspender la ejecución de una hebra es considerado obsoleto.
  \item Uso del namespace \textbf{chrono} para definir periodos de tiempo de forma flexible (precisión de segundos,minutos, horas, microsegundos, nanosegundos, etc.).
  \item Funciones \textbf{sleep\_for} y \textbf{sleep\_until} toman como argumento un \textbf{periodo de tiempo} y un \textbf{instante en el tiempo} respectivamente. El primero es un tiempo \textbf{relativo} al tiempo exacto en que se invoca a sleep\_for y el segundo es un tiempo \textbf{absoluto}, independiente del tiempo en que se invocó a sleep\_until.
\end{itemize}
\end{block}
\end{frame}

\begin{frame}[fragile]{Thread}
\begin{block}{Ejemplo de uso de sleep\_for}
\begin{lstlisting}[language=C++, basicstyle=\small]
#include <chrono>
#include <thread>

int main(int argc, char *argv[]) {
  std::chrono::milliseconds duration (5000); // 5 seconds
  std::this_thread::sleep_for(duration);
  
  return 0;
}
\end{lstlisting}
\end{block}
\end{frame}

\begin{frame}{Thread}
\begin{block}{Consideración con respecto a locks y similares}
\begin{itemize}
  \item Si se está trabajando con \textbf{variables de condición} o \textbf{locks}, el invocar a yield o a a algún tipo de sleep \textbf{no liberará el lock} del mutex que se había adquirido, por lo que se debe tener cuidado a la hora de llamar a estas funciones habiendo adquirido algún lock previamente.
\end{itemize}
\end{block}
\end{frame}
%
%\section{el header mutex}
%\include{mutex}

\section{Arquitecturas y modelos de computación paralela}
\begin{frame}
\vfill
\begin{center}
\begin{block}{\begin{center}\begin{Huge}Arquitecturas y modelos de computación paralela\end{Huge}\end{center}}
\end{block}
%\vfill
\end{center}
%\vfill
\end{frame}

\begin{frame}{Arquitecturas y modelos de computación paralela}
\begin{block}{Programación en GPU - Intro}
\begin{itemize}
  \item Historia de la GPU (ley de Moore). Aumento del número de unidades de procesamiento.
  \item Alternativa económica.
  \item Computación en CPU (orientados a ocultar la latencia) vs GPU (orientados al \textit{throughput}).
\end{itemize}
\centerline{\pgfimage[height=3.5cm]{pics/cpu_vs_gpu}}
\end{block}
\end{frame}

\begin{frame}{Arquitecturas y modelos de computación paralela}
\begin{block}{Programación en GPU - Modelo del dispositivo}
\centerline{\pgfimage[height=5.0cm]{pics/modelo_dispositivo}}
\end{block}
\end{frame}

\begin{frame}{Arquitecturas y modelos de computación paralela}
\begin{block}{Programación en GPU - Modelo de ejecución}
\begin{columns}

  \begin{column}{.50\textwidth}
	\centerline{
      \pgfimage[height=2.5cm]{pics/kernel_call}}
      %\pgfimage[height=3.5cm]{pics/house}}
  \end{column}
  
  \begin{column}{.50\textwidth}
    \centerline{
      \pgfimage[height=2.5cm]{pics/grid}}
      %\pgfimage[height=3.5cm]{pics/house}}
  \end{column}
  
\end{columns}
\end{block}
\end{frame}

\begin{frame}{Arquitecturas y modelos de computación paralela}
\begin{block}{Programación en GPU - Modelo de memoria}
\centerline{\pgfimage[height=5.0cm]{pics/modelo_memoria}}
\end{block}
\end{frame}

\begin{frame}{Arquitecturas y modelos de computación paralela}
\begin{block}{Programación en GPU - Consejos generales}
\begin{itemize}
  \item Kernels grandes, pocos accesos y gran cantidad de operaciones (ocultar latencia).
  \item Evitar colisiones en los accessos (uso de IDs).
  \item Bloques deben ser de tamaño múltiplo de \textit{warp-wavefront}.
  \item Mantener a la GPU ocupada (\textit{oversubscribe}).
  \item Evaluar uso de memoria compartida y memoria constante.
\end{itemize}
\end{block}
\end{frame}

\begin{frame}{Arquitecturas y modelos de computación paralela}
\begin{block}{SIMD y extensiones SSE}
\begin{itemize}
  \item Operaciones vectoriales haciendo uso de registros con mayor capacidad.
\end{itemize}
  \centerline{\pgfimage[height=4.5cm]{pics/simd}}
\end{block}
\end{frame}

\begin{frame}{Arquitecturas y modelos de computación paralela}
\begin{block}{SIMD y extensiones SSE - Evolución de la tecnología}
\begin{itemize}
  \item MMX.
  \item SSE (\textit{Streaming SIMD Extensions}).
  \item SSE2.
  \item SSE3.
  \item SSSE3.
  \item SSE4
  \item AVX.
  \item AVX512.
\end{itemize}
\end{block}
\end{frame}

\begin{frame}{Arquitecturas y modelos de computación paralela}
\begin{block}{SIMD y extensiones SSE - Empaquetamiento}
  \centerline{\pgfimage[height=4.5cm]{pics/avx_registers_type}}
\end{block}
\end{frame}

\section{Trabajo realizado}
\begin{frame}
\vfill
\begin{center}
\begin{block}{\begin{center}\begin{Huge}Trabajo realizado\end{Huge}\end{center}}
\end{block}
%\vfill
\end{center}
%\vfill
\end{frame}

%%%%%%%%%%%%%%%%%%%%%%%%%%%%%%%%%%%%%%%%%%%%%%%%%%%%%%%%%%%%%%%%%%%%%%%%%%%%
\begin{frame}{Trabajo realizado}
\centerline{
      \pgfimage[height=7.0cm]{pics/mc_algoritmo}}
\end{frame}

\begin{frame}{Trabajo realizado}
\begin{block}{Análisis de código y detección de zonas paralelizables}
\begin{itemize}
  \item Speedup alcanzable queda limitado por la ley de Amdahl.
  \item Primer bucle, encargado de construir una curva de histéresis por cada semilla, es trivialmente paralelizable.
  \item Segundo y tercer bucle (valor de campo y número de MCS) son secuenciales.
  \item Cuarto bucle, el encargado de tomar muestras al azar, es secuencial si de desea cumplir la condición de \textit{balance detallado}.
  \item Paralelización es solo aplicable al cálculo de la energía realizado en el bucle más interno.
\end{itemize}
\end{block}

\end{frame}

\begin{frame}{Trabajo realizado}
\begin{block}{Análisis de código y detección de zonas paralelizables}
\begin{itemize}
  \item El cálculo de la interacción dipolar corresponde a otro bucle anidado, ya que debe calcularse la interacción entre cada uno de los espines.
  \item De las mediciones realizadas, el simulador tarda un 99\% del tiempo total en calcular la interacción dipolar.
  \item Tal cálculo no es en sí computacionalmente costoso, pero éste debe ser realizado un gran número de veces.
\end{itemize}
\centerline{
      \pgfimage[height=3.0cm]{pics/patron}}
\end{block}

\end{frame}
%%%%%%%%%%%%%%%%%%%%%%%%%%%%%%%%%%%%%%%%%%%%%%%%%%%%%%%%%%%%%%%%%%%%%%%%%%%%
\begin{frame}{Trabajo realizado}
\begin{block}{Paralelización a través de OpenMP}
\begin{itemize}
  \item Primer intento: \textit{straightforward parallelism}.
  \item Basado en filosofía de OpenMP (paralelismo incremental).
  \item Creación y destrucción de hebras resulta ser ineficiente.
\end{itemize}
\end{block}
\end{frame}

\begin{frame}[fragile]{Trabajo realizado}
\begin{block}{Primer intento con OpenMP}

\begin{lstlisting}[language=C++, basicstyle=\tiny]
for (int j = 0; j < nr_deltaH; j++) {
  for (int k = 0; k < MCS; k++) {
    for (int l = 0; l < nr_spins; l++) {
      // seleccionar spin al azar

      // calcular dipolar
      #pragma omp parallel for
      for (int i = 0; i < nr_spins; i++) {
        // codigo dipolar
      }

      // suma valores parciales dipolar

      // calcular resto de las interacciones
      // y decidir si mantener o actualizar la configuracion
    }
  }
}
\end{lstlisting}

\end{block}
\end{frame}
%%%%%%%%%%%%%%%%%%%%%%%%%%%%%%%%%%%%%%%%%%%%%%%%%%%%%%%%%%%%%%%%%%%%%%%%%%%%
\begin{frame}{Trabajo realizado}
\begin{block}{Paralelización a través de OpenMP}
\begin{itemize}
  \item Segundo intento: mantener hebras activas.
  \item Por defecto, todo el código es ejecutado por todas las hebras.
  \item Se debe definir ahora zonas de códigos que serán ejecutadas por tan solo una hebra y agregar mecanismos de sincronización.
\end{itemize}
\end{block}
\end{frame}

\begin{frame}[fragile]{Trabajo realizado}
\begin{block}{Segundo intento con OpenMP}

\begin{lstlisting}[language=C++, basicstyle=\tiny]
#pragma omp parallel
{
for (int j = 0; j < nr_deltaH; j++) {
  for (int k = 0; k < MCS; k++) {
    for (int l = 0; l < nr_spins; l++) {
      #pragma omp single
      // seleccionar spin al azar

      // calcular dipolar
      calculateDipolar();

      #pragma omp critical
      // suma valores parciales dipolar

      #pragma omp barrier

      #pragma omp single
      // calcular resto de las interacciones
      // y decidir si mantener o actualizar la configuracion
    }
  }
}
}   // end pragma omp parallel
\end{lstlisting}

\end{block}
\end{frame}

%%%%%%%%%%%%%%%%%%%%%%%%%%%%%%%%%%%%%%%%%%%%%%%%%%%%%%%%%%%%%%%%%%%%%%%%%%%%
\begin{frame}{Trabajo realizado}
\begin{block}{Paralelización en GPU a través de CUDA y OpenCL}
\begin{itemize}
  \item No se logra implementar la misma estrategia de mantener las hebras activas ya que el resultado debe ser comunicado de vuelta al host y no existen mecanismos de sincronización entre hebras que perteneces a bloques distintos.
  \item Se debe transferir los valores de magnetización por cada ejecución del kernel.
  \item Para compensar el costo de comunicación, la cantidad de espines debe ser lo suficientemente grande como para que el cálculo de la interacción dipolar valga la pena ser calculado utilizando miles de hebras.
\end{itemize}
\end{block}
\end{frame}

\begin{frame}{Trabajo realizado}
\begin{block}{Paralelización en GPU a través de CUDA y OpenCL}
\begin{itemize}
  \item Tamaño del sistema está limitado por el tamaño de la memoria global.
  \item Cantidad de accesos a memoria deben ser idealmente menores a la cantidad de operaciones realizadas en el kernel.
\end{itemize}
\begin{equation}
\label{formula:kerneldipolar}
\begin{aligned}
  d_f &= 3(\hat{n_{ij}} \cdot \vec{m_j}) \\
  {B_d} &= \frac{d_f\hat{n_{ij}} - \vec{m_j}}{r_{ij}^3} \\
  %
\end{aligned}
\end{equation}
\end{block}
\end{frame}

\begin{frame}[fragile]{Trabajo realizado}
\begin{block}{Función kernel}

\begin{lstlisting}[language=C++, basicstyle=\tiny]
kernel_dipolar(float *magnetization, float *nx, float *ny, float *nz,
        float *cube, int site, int nrAtoms, float *output) {
    // index es el id de la hebra
    if (index >= nrAtoms)                   // 1 comparacion
        return;

    int inputIndexX = 0 * nrAtoms + index;  // 2 op aritmeticas
    int inputIndexY = 1 * nrAtoms + index;  // 2 op aritmeticas
    int inputIndexZ = 2 * nrAtoms + index;  // 2 op aritmeticas

    float mujX = magnetization[inputIndexX];
    float mujY = magnetization[inputIndexY];
    float mujZ = magnetization[inputIndexZ];

    int indexJSite = site * nrAtoms + index;    // 2 op aritmeticas

    float distanceX = nx[indexJSite];
    float distanceY = ny[indexJSite];
    float distanceZ = nz[indexJSite];

    ...
\end{lstlisting}

\end{block}
\end{frame}

\begin{frame}[fragile]{Trabajo realizado}
\begin{block}{Función kernel (continuación)}

\begin{lstlisting}[language=C++, basicstyle=\tiny]
	...
    float cubeDistance = cube[indexJSite];

    // 6 operaciones aritmeticas
    float df = 3 * (mujX*distanceX + mujY*distanceY + mujZ*distanceZ);

    output[inputIndexX] = (df * distanceX - mujX) / cubeDistance; // 3 op aritimeticas
    output[inputIndexY] = (df * distanceY - mujY) / cubeDistance; // 3 op aritimeticas
    output[inputIndexZ] = (df * distanceZ - mujZ) / cubeDistance; // 3 op aritimeticas
}
\end{lstlisting}

\end{block}
\end{frame}

\begin{frame}{Trabajo realizado}
\begin{block}{Paralelización en GPU a través de CUDA y OpenCL}
\begin{itemize}
  \item Kernel anterior tiene un ratio CGMA (\textit{Compute to Global Memory Access}) de dos, es decir, por cada dos accesos a memoria se realizan dos operaciones.
  \item Dada la natureleza del problema, tampoco resulta factible utilizar otros tipos de memoria como la memoria compartida, constante, etc.
  \item Información se encuentra almacenada de forma contigua con el fin de que hebras contiguas accedan a posiciones contiguas (se evitan colisiones).
  \item Tampoco existen divergencias en el kernel más allá del \textit{if} inicial que verifica si se trata de un id válido.
\end{itemize}
\end{block}
\end{frame}
%%%%%%%%%%%%%%%%%%%%%%%%%%%%%%%%%%%%%%%%%%%%%%%%%%%%%%%%%%%%%%%%%%%%%%%%%%%%

\begin{frame}{Trabajo realizado}
\begin{block}{Paralelización a través de OpenMP e instrucciones SIMD}
\begin{itemize}
  \item Además de dividir la carga de trabajos entre hebras, se realiza procesamiento vectorial por cada una de ellas.
  \item Se utiliza la extensión AVX que permite operar en registros de 256 bits.
  \item Además de ofrecer vectorización, se evita además accesos a memoria al procesar la información directamente en los registros del procesador.
  \item Se implementan dos programas, uno que utiliza precisión simple y otro que utiliza precisión doble.
  \item Información debe ser empaquetada de forma correcta para realizar transferencias óptimas a los registros SIMD y poder operar así de forma vectorial.
\end{itemize}
\end{block}
\end{frame}

\begin{frame}{Trabajo realizado}
\begin{block}{Cálculo del valor \textit{df}}
\centerline{
      \pgfimage[height=7.0cm]{pics/df}}
\end{block}
\end{frame}

\begin{frame}{Trabajo realizado}
\begin{block}{Paralelización a través de OpenMP e instrucciones SIMD}
\begin{itemize}
  \item Operaciones SIMD para el programa de precisión simple difieren de las del programa de precisión doble.
  \item Se agrega un \textit{overhead} al realizar las operaciones finales del cálculo de la interacción dipolar, al retornar los valores de este vector de forma contigua.
\end{itemize}
\end{block}
\end{frame}

\section{Experimentos y resultados}
\begin{frame}
\vfill
\begin{center}
\begin{block}{\begin{center}\begin{Huge}Experimentos y resultados\end{Huge}\end{center}}
\end{block}
%\vfill
\end{center}
%\vfill
\end{frame}

%%%%%%%%%%%%%%%%%%%%%%%%%%%%%%%%%%%%%%%%%%%%%%%%%%%%%%%%%%%%%%%%%%%%%%%%%%%%

\begin{frame}{Experimentos}

\begin{block}{Mediciones}
\begin{itemize}
  \item Mediciones fueron realizadas con \texttt{clock} para el programa secuencial (tiempo de procesador utilizado en cantidad de \textit{ticks} de reloj) y \texttt{omp\_get\_wtime} para los programas paralelos (tiempo de reloj de muralla).
\end{itemize}
\end{block}

\begin{block}{Conjunto de pruebas}
\begin{itemize}
  \item Programa original secuencial (\textit{double}) con y sin -O3.
  \item Programa paralelizado con OpenMP (\textit{float} y \textit{double}).
  \item Programas paralelizados en GPU utilizando CUDA y OpenCL.
  \item Programa paralelizado con OpenMP + SIMD (\textit{float} y \textit{double}).
\end{itemize}
\end{block}

\end{frame}

%%%%%%%%%%%%%%%%%%%%%%%%%%%%%%%%%%%%%%%%%%%%%%%%%%%%%%%%%%%%%%%%%%%%%%%%%%%%

\begin{frame}{Experimentos}
\begin{block}{Entorno de Pruebas: CPU}

\begin{table}[]
\centering
%\caption{My caption}
\label{my-label}
\begin{tabular}{lrr}
\toprule
           & \multicolumn{1}{c}{\bf Nanoserver}          & \multicolumn{1}{c}{\bf Titan}               \\
\midrule
Modelo     & \multicolumn{1}{l}{AMD Opteron 6282 SE} & \multicolumn{1}{l}{Intel Core i7-4960X} \\
Nr. cores  & 64 (4 x 16)                             & 6                                       \\
Frecuencia & 2.6 GHz                                 & 3.6 GHz      \\  
\bottomrule
\end{tabular}
\end{table}

\end{block}
\end{frame}


\begin{frame}{Experimentos}
\begin{block}{Entorno de Pruebas: Tarjeta gráfica}

\begin{table}[]
\centering
%\caption{My caption}
\label{my-label}
\begin{tabular}{lrr}
\toprule
               & \multicolumn{1}{c}{\textbf{Nanoserver}} & \multicolumn{1}{c}{\textbf{Titan}}     \\
\midrule
Modelo         & \multicolumn{1}{l}{Nvidia Tesla c2075}  & \multicolumn{1}{l}{Nvidia Titan Black} \\
RAM            & 6 GB                                    & 6 GB                                   \\
Bus de memoria & 384 bit                                 & 384 bit                                \\
Bandwidth      & 144 GB/s                                & 336 GB/s                               \\
Conexión       & PCIe 2.0x12                             & PCIe 3.0x16                            \\
Frecuencia GPU & 1.150 MHz                               & 889 MHz                                \\
Nr. shaders    & 448 (14 SM x 32 SP)                     & 2.880 (15 SM x 192 SP)                \\
\bottomrule
\end{tabular}
\end{table}

\end{block}
\end{frame}

%%%%%%%%%%%%%%%%%%%%%%%%%%%%%%%%%%%%%%%%%%%%%%%%%%%%%%%%%%%%%%%%%%%%%%%%%%%%
%%%%%%%%%%%%%%%%%%%%%%%%%%%%%%%%%%%%%%%%%%%%%%%%%%%%%%%%%%%%%%%%%%%%%%%%%%%%%%%%%%%%%%%%%
%%%%%%%%%%%%%%%%%%%%%%%%%%%%%%%%%%%%%%%%%%%%%%%%%%%%%%%%%%%%%%%%%%%%%%%%%%%%%%%%%%%%%%%%%
\begin{frame}{Experimentos}
\begin{block}{Tiempos de ejecución y speedup: CUDA y OpenCL}

\begin{columns}
  \begin{column}{.50\textwidth}
	\centerline{
      \pgfimage[height=4.5cm]{pics/graficosEs/titan_gpu}}
      %\pgfimage[height=3.5cm]{pics/house}}

  \end{column}
  \begin{column}{.50\textwidth}
    \centerline{
      \pgfimage[height=4.5cm]{pics/graficosEs/titan_gpu_speedup}}
      %\pgfimage[height=3.5cm]{pics/house}}
  \end{column}
  \end{columns}
\end{block}
\end{frame}

\begin{frame}{Experimentos}
\begin{block}{Comentarios: CUDA y OpenCL}
\begin{itemize}
	  \item Configuración utiliza el máximo número de hebras por bloques.
	  \item A partir de 3.000 espines, el tiempo de ejecución en GPU es menor al O3.
	  \item Speedup creciente dada la pendiente de la curva de speedup.
	  \item A mayor tamaño de problema, mayor es la transferencia CPU-GPU y mayor es la cantidad de veces a realizar tal comunicación.
	  \item Tiempos de transferencia corresponden en promedio a un 77\% (CUDA) y un 74\% (OpenCL) del total de tiempo de simulación.
	  \item Computación realizada en el kernel no justifica el alto costo de comunicación.
	\end{itemize}
\end{block}
\end{frame}
%%%%%%%%%%%%%%%%%%%%%%%%%%%%%%%%%%%%%%%%%%%%%%%%%%%%%%%%%%%%%%%%%%%%%%%%%%%%%%%%%%%%%%%%%%
%%%%%%%%%%%%%%%%%%%%%%%%%%%%%%%%%%%%%%%%%%%%%%%%%%%%%%%%%%%%%%%%%%%%%%%%%%%%%%%%%%%%%%%%%
\begin{frame}{Experimentos}
\begin{block}{Tiempos de ejecución: Titan OpenMP + SIMD}

\begin{columns}
  \begin{column}{.50\textwidth}
	\centerline{
      \pgfimage[height=4.5cm]{pics/graficosEs/titan_simd_double}}
  \end{column}
  \begin{column}{.50\textwidth}
    \centerline{
      \pgfimage[height=4.5cm]{pics/graficosEs/titan_simd_float}}
  \end{column}
\end{columns}
\end{block}
\end{frame}

\begin{frame}{Experimentos}
\begin{block}{Speedup: Titan OpenMP + SIMD}

\begin{columns}
  \begin{column}{.50\textwidth}
	\centerline{
      \pgfimage[height=4.5cm]{pics/graficosEs/titan_simd_double_speedup}}
      %\pgfimage[height=3.5cm]{pics/house}}

  \end{column}
  \begin{column}{.50\textwidth}
    \centerline{
      \pgfimage[height=4.5cm]{pics/graficosEs/titan_simd_float_speedup}}
      %\pgfimage[height=3.5cm]{pics/house}}
  \end{column}
  \end{columns}
\end{block}
\end{frame}
%%%%%%%%%%%%%%%%%%%%%%%%%%%%%%%%%%%%%%%%%%%%%%%%%%%%%%%%%%%%%%%%%%%%%%%%%%%%%%%%%%%%%%%%%
\begin{frame}{Experimentos}
\begin{block}{Comentarios: Mejores speedups obtenidos}

\begin{table}[htbp]\small
\centering
%\caption{Resumen speedups alcanzados}
\label{tabla:resumenspeedups}
\begin{tabular}{lll}
\toprule
\multicolumn{1}{c}{{\bf Programa}} & \multicolumn{1}{c}{{\bf Nanoserver}} & \multicolumn{1}{c}{{\bf Titan}} \\
\midrule
OpenMP double & 8.78 (4) & 9.01 (6) \\
OpenMP float & 13.67 (8) & 9.14 (6) \\
CUDA & sin pruebas & 4.98 (19 bloques x 1024 hebras) \\
OpenCL & sin pruebas & 3.13 (16 bloques x 1024 hebras) \\
OpenMP + SIMD double & 10.47 (4) & 15.21 (6) \\
OpenMP + SIMD float & 19.5 (4) & 30.65 (6) \\
\bottomrule
\end{tabular}
\end{table}

\end{block}
\end{frame}

\begin{frame}{Experimentos}
\begin{block}{Comentarios adicionales}

\begin{itemize}
  \item Speedups obtenidos en el Titan son por lo general mejores que los obtenidos en el Nanoserver.
  \item Pendiente de las curvas de speedups son mayores en el Nanoserver, por lo que se podría seguir explotando la mayor cantidad de cores que posee el Nanoserver al seguir incrementando la cantidad de espines.
  \item Speedups obtenidos en GPU son considerablemente más bajos que los obtenidos con los otros métodos de paralelización.
\end{itemize}
\end{block}
\end{frame}

\begin{frame}{Experimentos}
\begin{block}{Comentarios adicionales}

\begin{itemize}
  \item Paralelización en GPU también presenta el problema de contar con un menor tamaño de memoria.
  \item Computación en GPU requiere obviamente hardware adicional.
  \item tecnología SIMD resulta prometedora. Próxima generación de procesadores contarán con AVX512.
  \item De las pruebas realizadas, nunca se obtuvo un speedup lineal perfecto (siempre hay un costo por paralelización).
\end{itemize}
\end{block}
\end{frame}
%%%%%%%%%%%%%%%%%%%%%%%%%%%%%%%%%%%%%%%%%%%%%%%%%%%%%%%%%%%%%%%%%%%%%%%%%%%%%%%%%%%%%%%%%
\begin{frame}{Experimentos}
\begin{block}{Validación}

\begin{itemize}
  \item Uso del comando diff para buscar diferencias entre archivos.
  \item Programas que utilizan \texttt{doubles} producen exactamente los mismos resultados.
  \item Programas que utilizan \texttt{floats} producen resultados que difieren tan solo a partir de la quinta o sexta posición decimal.
  \item Valores originales corresponden a valores obtenidos a través de simulaciones.
\end{itemize}
\end{block}
\end{frame}

\section{Conclusiones}
\begin{frame}
\vfill
\begin{center}
\begin{block}{\begin{center}\begin{Huge}Conclusiones\end{Huge}\end{center}}
\end{block}
%\vfill
\end{center}
%\vfill
\end{frame}

\begin{frame}{Conclusiones}
\begin{itemize}
  \item Se investigó sobre técnicas de paralelismo en GPU y CPU y su aplicabilidad en el modelo de Heisenberg.
  \item Primera paralelización utilizando \textit{straightforward parallelism} con OpenMP no cumplió las expectativas.
  \item Se ofreció un segundo método de paralelización en OpenMP, el cual busca mantener las hebras activas. Este método presentó mejores resultados pero la estrategia utilizada no pudo ser portada a GPU.
  \item Dentro de las pruebas realizadas, se alcanzó un speedup de 30 utilizando OpenMP + SIMD, al simular 21.000 espines utilizando 6 hebras.
  \item Resultados obtenidos en GPU fueron considerablemente menores de lo esperado. Esto se debe tanto a la inclusión del término dipolar, la mayor cantidad de memoria necesitada en el modelo de Heisenberg frente a otros modelos, y a la arquitecturas de las GPUs.
\end{itemize}
\end{frame}

\begin{frame}{Conclusiones}
\begin{itemize}
  \item En GPU, resultados obtenidos utilizando CUDA son levemente mejores que los obtenidos con OpenCL.
  \item En cuanto a analisis de escalabilidad, se determinó que la solución OpenMP + SIMD es escalable y se espera alcanzar cifras mayores de speedups al simular instancias con mayor número de espines.
  \item Se determinó la ventaja de utilizar números flotantes de precisión simple y se verificó que los resultados obtenidos en estas simulaciones están en conformidad con los que se obtienen al utilizar \textit{doubles}.
  \item Se plantea como trabajo a futuro implementar el método de Monte Carlo en alguna FPGA, lo cual sería lo más cercano a construir hardware específico para tal simulación.
\end{itemize}
\end{frame}

\section{}
\begin{frame}
\vfill
%\begin{center}\begin{Huge}Vielen Dank \\[10pt]
%f\"ur ihre Aufmerksamkeit.\end{Huge}\vfill
%\end{center}
\begin{center}
\begin{Huge}
Consultas
\end{Huge}
%\vfill
\end{center}
%\vfill
% centerline centra el contenido horizontalmente
\centerline{\pgfimage[height=3.5cm]{pics/questions}}
\end{frame}

\end{document}
