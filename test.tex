\documentclass[xcolor=table]{beamer}
\usepackage[utf8]{inputenc}
\usepackage{listings}
\usepackage{xcolor} % for setting colors
\usepackage{booktabs}	% for tables toprule, midrule and bottomrule
%\usepackage[]{algorithm2e}
%\usepackage{algpseudocode}
%\usepackage{algorithm}
\usetheme{ZBH}

% set the default code style
\lstset{
	frame=tb, % draw a frame at the top and bottom of the code block
    tabsize=2, % tab space width
    showstringspaces=false, % don't mark spaces in strings
    numbers=left, % display line numbers on the left
    commentstyle=\color{green}, % comment color
    keywordstyle=\color{blue}, % keyword color
    stringstyle=\color{orange} % string color
}

\author{Sebastián Vizcay}
\title{Simulación de Monte Carlo paralela para sistemas ferromagnéticos en un modelo de Heisenberg incluyendo interacciones dipolares}
\date{\today}

\begin{document}

\begin{frame}

\titlepage
\end{frame}

\begin{frame}{Paralelización de simulación de Monte Carlo}
\begin{block}{Tabla de contenidos}
\begin{itemize}
  \item Introducción.
  \item Marco teórico.
  \item Arquitecturas y modelos de computación paralela.
  \item Trabajo realizado.
  \item Experimentos y resultados.
  \item Conclusiones.
\end{itemize}
\end{block}
%	\tableofcontents
\end{frame}

\section{Introducción}
\begin{frame}
\vfill
\begin{center}
\begin{block}{\begin{center}\begin{Huge}Introducción\end{Huge}\end{center}}
\end{block}
%\vfill
\end{center}
%\vfill
\end{frame}

\begin{frame}{Introducción}
\begin{block}{Antecedentes y motivación}
\begin{itemize}
  \item Avances en la ciencia han permitido la sintetización de nanopartículas (otorgándole diversas formas y tamaños).
  \item Propiedades físicas de nanoestructuras difieren de las macroscópias, las cuales son bien conocidas.
  \item Estructuras ferromagnéticas a escala nanométrica presentan propiedades tales como: magneto resistencia gigante, efecto Hall gigante, entre otras.
  \item Interés en el estudio de estas propiedades emergentes.
  \item Propiedades se encuentran determinadas por la geometría, tamaño y material.
  \item Búsqueda de nuevas geometrías con aplicaciones interesantes o con fines teóricos.
\end{itemize}
\end{block}
\end{frame}

\begin{frame}{Introducción}
\begin{block}{Geometrías convencionales}
\centerline{\pgfimage[height=4.5cm]{pics/geometrias}}
\end{block}
\end{frame}

\begin{frame}{Introducción}
\begin{block}{Antecedentes y motivación}

\begin{columns}

  \begin{column}{.65\textwidth}
\begin{itemize}
  \item Investigadores del Depto de Física de la Universidad de Santiago en conjunto con CEDENNA realizan estudio sistemático de estas estructuras a través de simulaciones de Monte Carlo.
  \item Aplicaciones van desde almacenamiento de información de alta densidad hasta portadores para entrega de drogras.
\end{itemize}
  \end{column}
  
  \begin{column}{.35\textwidth}
\centerline{\pgfimage[height=4.5cm]{pics/dds}}
  \end{column}
  
\end{columns}

\end{block}
\end{frame}

\begin{frame}{Introducción}
\begin{block}{Problemas}
\begin{itemize}
  \item Método de Monte Carlo es el método clásico en la Física Estadística para evaluar observables.
  \item Poder de cómputo actual no hace factible simular geometrías con más de $10^{9}$ momentos magnéticos.
  \item Se utiliza el método \textit{Fast Monte Carlo} para simular sistemas escalados.
  \item Tiempo de ejecución todavía es demasiado (orden de dos o tres semanas para 3.000 momentos magnéticos).
  \item Si se acelera el proceso, se podría estudiar sistemas más grandes o simplemente reducir los tiempos de simulación para proceder a analizar si la geometría es útil o no.
\end{itemize}
\end{block}
\end{frame}

\begin{frame}{Introducción}
\begin{block}{Solución y objetivo}
\begin{itemize}
  \item Se propone el diseño e implementación de un programa paralelo equivalente al secuencial actual que reduzca los tiempos de ejecución.
\end{itemize}
\end{block}
\begin{block}{Objetivos específicos}
\begin{itemize}
  \item Determinar método de paralelismo (GPU/CPU).
  \item Implementar software.
  \item Validar resultados (que sean correctos).
  \item Proponer experimentos para evaluar rendimiento computacional.
  \item Realizar las pruebas de ejecución.
\end{itemize}
\end{block}
\end{frame}

\section{Marco teórico}
\begin{frame}
\vfill
\begin{center}
\begin{block}{\begin{center}\begin{Huge}Marco teórico\end{Huge}\end{center}}
\end{block}
%\vfill
\end{center}
%\vfill
\end{frame}

\begin{frame}{Marco teórico}
\begin{block}{Modelos de espines}
\begin{itemize}
  \item Momento magnético: vector compuesto del momento angular del átomo y de su momento angular intrínseco o espín.
  \item Materiales pueden ser clasificados acorde a como sus espines reaccionan a un campo magnético externo.
  \item Materiales ferromagnéticos quedan permanentemente magnetizados al ser expuestos.
\end{itemize}
\end{block}
\end{frame}

\begin{frame}{Marco teórico}
\begin{block}{Modelos de espines - Estados fundamentales}
\begin{itemize}
  \item Modificación de las estructuras ferromagnéticas, como respuesta a un campo externo, genera diferentes estados.
\end{itemize}
\centerline{\pgfimage[height=4.5cm]{pics/configuraciones}}
\end{block}
\end{frame}

\begin{frame}{Marco teórico}
\begin{block}{Modelos de espines - Histéresis (modo de reversión)}
\begin{itemize}
  \item Forma en que el sistema recorre diferentes valores de campo desde la saturación en una dirección opuesta.
\end{itemize}

\begin{columns}

  \begin{column}{.50\textwidth}
\begin{itemize}
  \item Coercitividad: intensidad necesaria para eliminar magnetización.
  \item Magnetización de remanencia.
\end{itemize}
  \end{column}
  
  \begin{column}{.50\textwidth}
	\centerline{\pgfimage[height=4.0cm]{pics/histeresis}}
  \end{column}
  
\end{columns}




\end{block}
\end{frame}

\begin{frame}{Marco teórico}
\begin{block}{Modelos de espines}
\begin{itemize}
  \item Uso de modelos de grilla de espines (\textit{lattice spin models}).
  \item Modelos son descritos a través de un Hamiltoniano (función que determina la cantidad de energía del sistema).
  \item En este trabajo se utiliza el modelo de Heisenberg.
\end{itemize}
\end{block}
\end{frame}

\begin{frame}{Marco teórico}
\begin{block}{Modelos de espines - Energías}
\begin{itemize}
  \item Energía Dipolar.
  \item Energía de Intercambio (vecinos).
  \item Energía de Zeeman (campo externo y espín).
  \item Energía de anisotropía (dirección preferencial).
\end{itemize}

\begin{equation}
\label{formula:hamiltoniano}
    H = \frac{1}{2} \sum\limits_{i\neq j} [E_{ij}^{dip} - E_{ij}^{ex}] + \sum\limits_{i} E_i^k + \sum\limits_{i} E_i^z
\end{equation}

\end{block}
\end{frame}

\begin{frame}{Marco teórico}
\begin{block}{Simulaciones de Monte Carlo}
\begin{itemize}
  \item Simulaciones de los modelos anteriores son realizadas a través del método de Monte Carlo.
  \item Técnica matemática computarizada que hace uso de números aleatorios para simular procesos estocásticos.
  \item Trayectoria estocástica en el espacio de fase.

\end{itemize}
\begin{equation}
\label{formula:estimacionmontecarlo}
    X_{MC} = \frac{1}{M}\sum\limits_{i=1}^{M} f(x_i)
\end{equation}
\end{block}
\end{frame}

\begin{frame}{Marco teórico}
\begin{block}{Markov Chain Monte Carlo}
\begin{itemize}
  \item Uso de cadenas de Markov para samplear estados con mayor probabilidad.
  \item En base a si se cumplen ciertas propiedades, la cadena converge a una distribución estacionaria.

\end{itemize}
	\centerline{\pgfimage[height=4.0cm]{pics/markov}}
\end{block}
\end{frame}

\begin{frame}{Marco teórico}
\begin{block}{Algoritmo de Metrópolis}
\begin{itemize}
  \item Es un caso particular de \textit{Markov Chain} Monte Carlo en donde se descompone la probabilidad de transición en una probabilidad de selección un estado y luego aceptar tal transación.
  \item Lo anterior permite realizar una simplificación en donde se elimina la constante de normalización (la cual no puede ser calculada debido a la infinidad de estados posibles).

\end{itemize}
	\centerline{\pgfimage[height=3.5cm]{pics/metropolis}}
\end{block}
\end{frame}

\begin{frame}{Marco teórico}
\begin{block}{Técnicas para acelerar Monte Carlo}
\begin{itemize}
  \item Métodos de clustering: Swendsen-Wang, Wolff (utilizados en sistemas con temperatura cercana a la temperatura crítica).
  \item Parallel tempering (utilizado en sistemas en criticalidad y en modelo \textit{Spin Glass}).
  \item Descomposición en tablero de ajedrez (no considera dipolar ni respeta el balance detallado).
  \item Se opta por mantener el método de actualización secuencial (\textit{single spin flip}).
  \item Se optimiza el cálculo del Hamiltoniano, calculando el delta de energía solamente en base al espín seleccionado de forma aleatoria.
\end{itemize}
\end{block}
\end{frame}

\section{Arquitecturas y modelos de computación paralela}
\begin{frame}
\vfill
\begin{center}
\begin{block}{\begin{center}\begin{Huge}Arquitecturas y modelos de computación paralela\end{Huge}\end{center}}
\end{block}
%\vfill
\end{center}
%\vfill
\end{frame}

\begin{frame}{Arquitecturas y modelos de computación paralela}
\begin{block}{Programación en GPU - Intro}
\begin{itemize}
  \item Historia de la GPU (ley de Moore). Aumento del número de unidades de procesamiento.
  \item Alternativa económica.
  \item Computación en CPU (orientados a ocultar la latencia) vs GPU (orientados al \textit{throughput}).
\end{itemize}
\centerline{\pgfimage[height=3.5cm]{pics/cpu_vs_gpu}}
\end{block}
\end{frame}

\begin{frame}{Arquitecturas y modelos de computación paralela}
\begin{block}{Programación en GPU - Modelo del dispositivo}
\centerline{\pgfimage[height=5.0cm]{pics/modelo_dispositivo}}
\end{block}
\end{frame}

\begin{frame}{Arquitecturas y modelos de computación paralela}
\begin{block}{Programación en GPU - Modelo de ejecución}
\begin{columns}

  \begin{column}{.50\textwidth}
	\centerline{
      \pgfimage[height=2.5cm]{pics/kernel_call}}
      %\pgfimage[height=3.5cm]{pics/house}}
  \end{column}
  
  \begin{column}{.50\textwidth}
    \centerline{
      \pgfimage[height=2.5cm]{pics/grid}}
      %\pgfimage[height=3.5cm]{pics/house}}
  \end{column}
  
\end{columns}
\end{block}
\end{frame}

\begin{frame}{Arquitecturas y modelos de computación paralela}
\begin{block}{Programación en GPU - Modelo de memoria}
\centerline{\pgfimage[height=5.0cm]{pics/modelo_memoria}}
\end{block}
\end{frame}

\begin{frame}{Arquitecturas y modelos de computación paralela}
\begin{block}{Programación en GPU - Consejos generales}
\begin{itemize}
  \item Kernels grandes, pocos accesos y gran cantidad de operaciones (ocultar latencia).
  \item Evitar colisiones en los accessos (uso de IDs).
  \item Bloques deben ser de tamaño múltiplo de \textit{warp-wavefront}.
  \item Mantener a la GPU ocupada (\textit{oversubscribe}).
  \item Evaluar uso de memoria compartida y memoria constante.
\end{itemize}
\end{block}
\end{frame}

\begin{frame}{Arquitecturas y modelos de computación paralela}
\begin{block}{SIMD y extensiones SSE}
\begin{itemize}
  \item Operaciones vectoriales haciendo uso de registros con mayor capacidad.
\end{itemize}
  \centerline{\pgfimage[height=4.5cm]{pics/simd}}
\end{block}
\end{frame}

\begin{frame}{Arquitecturas y modelos de computación paralela}
\begin{block}{SIMD y extensiones SSE - Evolución de la tecnología}
\begin{itemize}
  \item MMX.
  \item SSE (\textit{Streaming SIMD Extensions}).
  \item SSE2.
  \item SSE3.
  \item SSSE3.
  \item SSE4
  \item AVX.
  \item AVX512.
\end{itemize}
\end{block}
\end{frame}

\begin{frame}{Arquitecturas y modelos de computación paralela}
\begin{block}{SIMD y extensiones SSE - Empaquetamiento}
  \centerline{\pgfimage[height=4.5cm]{pics/avx_registers_type}}
\end{block}
\end{frame}

\section{Trabajo realizado}
\begin{frame}
\vfill
\begin{center}
\begin{block}{\begin{center}\begin{Huge}Trabajo realizado\end{Huge}\end{center}}
\end{block}
%\vfill
\end{center}
%\vfill
\end{frame}

%%%%%%%%%%%%%%%%%%%%%%%%%%%%%%%%%%%%%%%%%%%%%%%%%%%%%%%%%%%%%%%%%%%%%%%%%%%%
\begin{frame}{Trabajo realizado}
\centerline{
      \pgfimage[height=7.0cm]{pics/mc_algoritmo}}
\end{frame}

\begin{frame}{Trabajo realizado}
\begin{block}{Análisis de código y detección de zonas paralelizables}
\begin{itemize}
  \item Speedup alcanzable queda limitado por la ley de Amdahl.
  \item Primer bucle, encargado de construir una curva de histéresis por cada semilla, es trivialmente paralelizable.
  \item Segundo y tercer bucle (valor de campo y número de MCS) son secuenciales.
  \item Cuarto bucle, el encargado de tomar muestras al azar, es secuencial si de desea cumplir la condición de \textit{balance detallado}.
  \item Paralelización es solo aplicable al cálculo de la energía realizado en el bucle más interno.
\end{itemize}
\end{block}

\end{frame}

\begin{frame}{Trabajo realizado}
\begin{block}{Análisis de código y detección de zonas paralelizables}
\begin{itemize}
  \item El cálculo de la interacción dipolar corresponde a otro bucle anidado, ya que debe calcularse la interacción entre cada uno de los espines.
  \item De las mediciones realizadas, el simulador tarda un 99\% del tiempo total en calcular la interacción dipolar.
  \item Tal cálculo no es en sí computacionalmente costoso, pero éste debe ser realizado un gran número de veces.
\end{itemize}
\centerline{
      \pgfimage[height=3.0cm]{pics/patron}}
\end{block}

\end{frame}
%%%%%%%%%%%%%%%%%%%%%%%%%%%%%%%%%%%%%%%%%%%%%%%%%%%%%%%%%%%%%%%%%%%%%%%%%%%%
\begin{frame}{Trabajo realizado}
\begin{block}{Paralelización a través de OpenMP}
\begin{itemize}
  \item Primer intento: \textit{straightforward parallelism}.
  \item Basado en filosofía de OpenMP (paralelismo incremental).
  \item Creación y destrucción de hebras resulta ser ineficiente.
\end{itemize}
\end{block}
\end{frame}

\begin{frame}[fragile]{Trabajo realizado}
\begin{block}{Primer intento con OpenMP}

\begin{lstlisting}[language=C++, basicstyle=\tiny]
for (int j = 0; j < nr_deltaH; j++) {
  for (int k = 0; k < MCS; k++) {
    for (int l = 0; l < nr_spins; l++) {
      // seleccionar spin al azar

      // calcular dipolar
      #pragma omp parallel for
      for (int i = 0; i < nr_spins; i++) {
        // codigo dipolar
      }

      // suma valores parciales dipolar

      // calcular resto de las interacciones
      // y decidir si mantener o actualizar la configuracion
    }
  }
}
\end{lstlisting}

\end{block}
\end{frame}
%%%%%%%%%%%%%%%%%%%%%%%%%%%%%%%%%%%%%%%%%%%%%%%%%%%%%%%%%%%%%%%%%%%%%%%%%%%%
\begin{frame}{Trabajo realizado}
\begin{block}{Paralelización a través de OpenMP}
\begin{itemize}
  \item Segundo intento: mantener hebras activas.
  \item Por defecto, todo el código es ejecutado por todas las hebras.
  \item Se debe definir ahora zonas de códigos que serán ejecutadas por tan solo una hebra y agregar mecanismos de sincronización.
\end{itemize}
\end{block}
\end{frame}

\begin{frame}[fragile]{Trabajo realizado}
\begin{block}{Segundo intento con OpenMP}

\begin{lstlisting}[language=C++, basicstyle=\tiny]
#pragma omp parallel
{
for (int j = 0; j < nr_deltaH; j++) {
  for (int k = 0; k < MCS; k++) {
    for (int l = 0; l < nr_spins; l++) {
      #pragma omp single
      // seleccionar spin al azar

      // calcular dipolar
      calculateDipolar();

      #pragma omp critical
      // suma valores parciales dipolar

      #pragma omp barrier

      #pragma omp single
      // calcular resto de las interacciones
      // y decidir si mantener o actualizar la configuracion
    }
  }
}
}   // end pragma omp parallel
\end{lstlisting}

\end{block}
\end{frame}

%%%%%%%%%%%%%%%%%%%%%%%%%%%%%%%%%%%%%%%%%%%%%%%%%%%%%%%%%%%%%%%%%%%%%%%%%%%%
\begin{frame}{Trabajo realizado}
\begin{block}{Paralelización en GPU a través de CUDA y OpenCL}
\begin{itemize}
  \item No se logra implementar la misma estrategia de mantener las hebras activas ya que el resultado debe ser comunicado de vuelta al host y no existen mecanismos de sincronización entre hebras que perteneces a bloques distintos.
  \item Se debe transferir los valores de magnetización por cada ejecución del kernel.
  \item Para compensar el costo de comunicación, la cantidad de espines debe ser lo suficientemente grande como para que el cálculo de la interacción dipolar valga la pena ser calculado utilizando miles de hebras.
\end{itemize}
\end{block}
\end{frame}

\begin{frame}{Trabajo realizado}
\begin{block}{Paralelización en GPU a través de CUDA y OpenCL}
\begin{itemize}
  \item Tamaño del sistema está limitado por el tamaño de la memoria global.
  \item Cantidad de accesos a memoria deben ser idealmente menores a la cantidad de operaciones realizadas en el kernel.
\end{itemize}
\begin{equation}
\label{formula:kerneldipolar}
\begin{aligned}
  d_f &= 3(\hat{n_{ij}} \cdot \vec{m_j}) \\
  {B_d} &= \frac{d_f\hat{n_{ij}} - \vec{m_j}}{r_{ij}^3} \\
  %
\end{aligned}
\end{equation}
\end{block}
\end{frame}

\begin{frame}[fragile]{Trabajo realizado}
\begin{block}{Función kernel}

\begin{lstlisting}[language=C++, basicstyle=\tiny]
kernel_dipolar(float *magnetization, float *nx, float *ny, float *nz,
        float *cube, int site, int nrAtoms, float *output) {
    // index es el id de la hebra
    if (index >= nrAtoms)                   // 1 comparacion
        return;

    int inputIndexX = 0 * nrAtoms + index;  // 2 op aritmeticas
    int inputIndexY = 1 * nrAtoms + index;  // 2 op aritmeticas
    int inputIndexZ = 2 * nrAtoms + index;  // 2 op aritmeticas

    float mujX = magnetization[inputIndexX];
    float mujY = magnetization[inputIndexY];
    float mujZ = magnetization[inputIndexZ];

    int indexJSite = site * nrAtoms + index;    // 2 op aritmeticas

    float distanceX = nx[indexJSite];
    float distanceY = ny[indexJSite];
    float distanceZ = nz[indexJSite];

    ...
\end{lstlisting}

\end{block}
\end{frame}

\begin{frame}[fragile]{Trabajo realizado}
\begin{block}{Función kernel (continuación)}

\begin{lstlisting}[language=C++, basicstyle=\tiny]
	...
    float cubeDistance = cube[indexJSite];

    // 6 operaciones aritmeticas
    float df = 3 * (mujX*distanceX + mujY*distanceY + mujZ*distanceZ);

    output[inputIndexX] = (df * distanceX - mujX) / cubeDistance; // 3 op aritimeticas
    output[inputIndexY] = (df * distanceY - mujY) / cubeDistance; // 3 op aritimeticas
    output[inputIndexZ] = (df * distanceZ - mujZ) / cubeDistance; // 3 op aritimeticas
}
\end{lstlisting}

\end{block}
\end{frame}

\begin{frame}{Trabajo realizado}
\begin{block}{Paralelización en GPU a través de CUDA y OpenCL}
\begin{itemize}
  \item Kernel anterior tiene un ratio CGMA (\textit{Compute to Global Memory Access}) de dos, es decir, por cada dos accesos a memoria se realizan dos operaciones.
  \item Dada la natureleza del problema, tampoco resulta factible utilizar otros tipos de memoria como la memoria compartida, constante, etc.
  \item Información se encuentra almacenada de forma contigua con el fin de que hebras contiguas accedan a posiciones contiguas (se evitan colisiones).
  \item Tampoco existen divergencias en el kernel más allá del \textit{if} inicial que verifica si se trata de un id válido.
\end{itemize}
\end{block}
\end{frame}
%%%%%%%%%%%%%%%%%%%%%%%%%%%%%%%%%%%%%%%%%%%%%%%%%%%%%%%%%%%%%%%%%%%%%%%%%%%%

\begin{frame}{Trabajo realizado}
\begin{block}{Paralelización a través de OpenMP e instrucciones SIMD}
\begin{itemize}
  \item Además de dividir la carga de trabajos entre hebras, se realiza procesamiento vectorial por cada una de ellas.
  \item Se utiliza la extensión AVX que permite operar en registros de 256 bits.
  \item Además de ofrecer vectorización, se evita además accesos a memoria al procesar la información directamente en los registros del procesador.
  \item Se implementan dos programas, uno que utiliza precisión simple y otro que utiliza precisión doble.
  \item Información debe ser empaquetada de forma correcta para realizar transferencias óptimas a los registros SIMD y poder operar así de forma vectorial.
\end{itemize}
\end{block}
\end{frame}

\begin{frame}{Trabajo realizado}
\begin{block}{Cálculo del valor \textit{df}}
\centerline{
      \pgfimage[height=7.0cm]{pics/df}}
\end{block}
\end{frame}

\begin{frame}{Trabajo realizado}
\begin{block}{Paralelización a través de OpenMP e instrucciones SIMD}
\begin{itemize}
  \item Operaciones SIMD para el programa de precisión simple difieren de las del programa de precisión doble.
  \item Se agrega un \textit{overhead} al realizar las operaciones finales del cálculo de la interacción dipolar, al retornar los valores de este vector de forma contigua.
\end{itemize}
\end{block}
\end{frame}

\section{Experimentos y resultados}
\begin{frame}
\vfill
\begin{center}
\begin{block}{\begin{center}\begin{Huge}Experimentos y resultados\end{Huge}\end{center}}
\end{block}
%\vfill
\end{center}
%\vfill
\end{frame}

%%%%%%%%%%%%%%%%%%%%%%%%%%%%%%%%%%%%%%%%%%%%%%%%%%%%%%%%%%%%%%%%%%%%%%%%%%%%

\begin{frame}{Experimentos}

\begin{block}{Mediciones}
\begin{itemize}
  \item Mediciones fueron realizadas con \texttt{clock} para el programa secuencial (tiempo de procesador utilizado en cantidad de \textit{ticks} de reloj) y \texttt{omp\_get\_wtime} para los programas paralelos (tiempo de reloj de muralla).
\end{itemize}
\end{block}

\begin{block}{Conjunto de pruebas}
\begin{itemize}
  \item Programa original secuencial (\textit{double}) con y sin -O3.
  \item Programa paralelizado con OpenMP (\textit{float} y \textit{double}).
  \item Programas paralelizados en GPU utilizando CUDA y OpenCL.
  \item Programa paralelizado con OpenMP + SIMD (\textit{float} y \textit{double}).
\end{itemize}
\end{block}

\end{frame}

%%%%%%%%%%%%%%%%%%%%%%%%%%%%%%%%%%%%%%%%%%%%%%%%%%%%%%%%%%%%%%%%%%%%%%%%%%%%

\begin{frame}{Experimentos}
\begin{block}{Entorno de Pruebas: CPU}

\begin{table}[]
\centering
%\caption{My caption}
\label{my-label}
\begin{tabular}{lrr}
\toprule
           & \multicolumn{1}{c}{\bf Nanoserver}          & \multicolumn{1}{c}{\bf Titan}               \\
\midrule
Modelo     & \multicolumn{1}{l}{AMD Opteron 6282 SE} & \multicolumn{1}{l}{Intel Core i7-4960X} \\
Nr. cores  & 64 (4 x 16)                             & 6                                       \\
Frecuencia & 2.6 GHz                                 & 3.6 GHz      \\  
\bottomrule
\end{tabular}
\end{table}

\end{block}
\end{frame}


\begin{frame}{Experimentos}
\begin{block}{Entorno de Pruebas: Tarjeta gráfica}

\begin{table}[]
\centering
%\caption{My caption}
\label{my-label}
\begin{tabular}{lrr}
\toprule
               & \multicolumn{1}{c}{\textbf{Nanoserver}} & \multicolumn{1}{c}{\textbf{Titan}}     \\
\midrule
Modelo         & \multicolumn{1}{l}{Nvidia Tesla c2075}  & \multicolumn{1}{l}{Nvidia Titan Black} \\
RAM            & 6 GB                                    & 6 GB                                   \\
Bus de memoria & 384 bit                                 & 384 bit                                \\
Bandwidth      & 144 GB/s                                & 336 GB/s                               \\
Conexión       & PCIe 2.0x12                             & PCIe 3.0x16                            \\
Frecuencia GPU & 1.150 MHz                               & 889 MHz                                \\
Nr. shaders    & 448 (14 SM x 32 SP)                     & 2.880 (15 SM x 192 SP)                \\
\bottomrule
\end{tabular}
\end{table}

\end{block}
\end{frame}

%%%%%%%%%%%%%%%%%%%%%%%%%%%%%%%%%%%%%%%%%%%%%%%%%%%%%%%%%%%%%%%%%%%%%%%%%%%%
%%%%%%%%%%%%%%%%%%%%%%%%%%%%%%%%%%%%%%%%%%%%%%%%%%%%%%%%%%%%%%%%%%%%%%%%%%%%%%%%%%%%%%%%%
%%%%%%%%%%%%%%%%%%%%%%%%%%%%%%%%%%%%%%%%%%%%%%%%%%%%%%%%%%%%%%%%%%%%%%%%%%%%%%%%%%%%%%%%%
\begin{frame}{Experimentos}
\begin{block}{Tiempos de ejecución y speedup: CUDA y OpenCL}

\begin{columns}
  \begin{column}{.50\textwidth}
	\centerline{
      \pgfimage[height=4.5cm]{pics/graficosEs/titan_gpu}}
      %\pgfimage[height=3.5cm]{pics/house}}

  \end{column}
  \begin{column}{.50\textwidth}
    \centerline{
      \pgfimage[height=4.5cm]{pics/graficosEs/titan_gpu_speedup}}
      %\pgfimage[height=3.5cm]{pics/house}}
  \end{column}
  \end{columns}
\end{block}
\end{frame}

\begin{frame}{Experimentos}
\begin{block}{Comentarios: CUDA y OpenCL}
\begin{itemize}
	  \item Configuración utiliza el máximo número de hebras por bloques.
	  \item A partir de 3.000 espines, el tiempo de ejecución en GPU es menor al O3.
	  \item Speedup creciente dada la pendiente de la curva de speedup.
	  \item A mayor tamaño de problema, mayor es la transferencia CPU-GPU y mayor es la cantidad de veces a realizar tal comunicación.
	  \item Tiempos de transferencia corresponden en promedio a un 77\% (CUDA) y un 74\% (OpenCL) del total de tiempo de simulación.
	  \item Computación realizada en el kernel no justifica el alto costo de comunicación.
	\end{itemize}
\end{block}
\end{frame}
%%%%%%%%%%%%%%%%%%%%%%%%%%%%%%%%%%%%%%%%%%%%%%%%%%%%%%%%%%%%%%%%%%%%%%%%%%%%%%%%%%%%%%%%%%
%%%%%%%%%%%%%%%%%%%%%%%%%%%%%%%%%%%%%%%%%%%%%%%%%%%%%%%%%%%%%%%%%%%%%%%%%%%%%%%%%%%%%%%%%
\begin{frame}{Experimentos}
\begin{block}{Tiempos de ejecución: Titan OpenMP + SIMD}

\begin{columns}
  \begin{column}{.50\textwidth}
	\centerline{
      \pgfimage[height=4.5cm]{pics/graficosEs/titan_simd_double}}
  \end{column}
  \begin{column}{.50\textwidth}
    \centerline{
      \pgfimage[height=4.5cm]{pics/graficosEs/titan_simd_float}}
  \end{column}
\end{columns}
\end{block}
\end{frame}

\begin{frame}{Experimentos}
\begin{block}{Speedup: Titan OpenMP + SIMD}

\begin{columns}
  \begin{column}{.50\textwidth}
	\centerline{
      \pgfimage[height=4.5cm]{pics/graficosEs/titan_simd_double_speedup}}
      %\pgfimage[height=3.5cm]{pics/house}}

  \end{column}
  \begin{column}{.50\textwidth}
    \centerline{
      \pgfimage[height=4.5cm]{pics/graficosEs/titan_simd_float_speedup}}
      %\pgfimage[height=3.5cm]{pics/house}}
  \end{column}
  \end{columns}
\end{block}
\end{frame}
%%%%%%%%%%%%%%%%%%%%%%%%%%%%%%%%%%%%%%%%%%%%%%%%%%%%%%%%%%%%%%%%%%%%%%%%%%%%%%%%%%%%%%%%%
\begin{frame}{Experimentos}
\begin{block}{Comentarios: Mejores speedups obtenidos}

\begin{table}[htbp]\small
\centering
%\caption{Resumen speedups alcanzados}
\label{tabla:resumenspeedups}
\begin{tabular}{lll}
\toprule
\multicolumn{1}{c}{{\bf Programa}} & \multicolumn{1}{c}{{\bf Nanoserver}} & \multicolumn{1}{c}{{\bf Titan}} \\
\midrule
OpenMP double & 8.78 (4) & 9.01 (6) \\
OpenMP float & 13.67 (8) & 9.14 (6) \\
CUDA & sin pruebas & 4.98 (19 bloques x 1024 hebras) \\
OpenCL & sin pruebas & 3.13 (16 bloques x 1024 hebras) \\
OpenMP + SIMD double & 10.47 (4) & 15.21 (6) \\
OpenMP + SIMD float & 19.5 (4) & 30.65 (6) \\
\bottomrule
\end{tabular}
\end{table}

\end{block}
\end{frame}

\begin{frame}{Experimentos}
\begin{block}{Comentarios adicionales}

\begin{itemize}
  \item Speedups obtenidos en el Titan son por lo general mejores que los obtenidos en el Nanoserver.
  \item Pendiente de las curvas de speedups son mayores en el Nanoserver, por lo que se podría seguir explotando la mayor cantidad de cores que posee el Nanoserver al seguir incrementando la cantidad de espines.
  \item Speedups obtenidos en GPU son considerablemente más bajos que los obtenidos con los otros métodos de paralelización.
\end{itemize}
\end{block}
\end{frame}

\begin{frame}{Experimentos}
\begin{block}{Comentarios adicionales}

\begin{itemize}
  \item Paralelización en GPU también presenta el problema de contar con un menor tamaño de memoria.
  \item Computación en GPU requiere obviamente hardware adicional.
  \item tecnología SIMD resulta prometedora. Próxima generación de procesadores contarán con AVX512.
  \item De las pruebas realizadas, nunca se obtuvo un speedup lineal perfecto (siempre hay un costo por paralelización).
\end{itemize}
\end{block}
\end{frame}
%%%%%%%%%%%%%%%%%%%%%%%%%%%%%%%%%%%%%%%%%%%%%%%%%%%%%%%%%%%%%%%%%%%%%%%%%%%%%%%%%%%%%%%%%
\begin{frame}{Experimentos}
\begin{block}{Validación}

\begin{itemize}
  \item Uso del comando diff para buscar diferencias entre archivos.
  \item Programas que utilizan \texttt{doubles} producen exactamente los mismos resultados.
  \item Programas que utilizan \texttt{floats} producen resultados que difieren tan solo a partir de la quinta o sexta posición decimal.
  \item Valores originales corresponden a valores obtenidos a través de simulaciones.
\end{itemize}
\end{block}
\end{frame}

\section{Conclusiones}
\begin{frame}
\vfill
\begin{center}
\begin{block}{\begin{center}\begin{Huge}Conclusiones\end{Huge}\end{center}}
\end{block}
%\vfill
\end{center}
%\vfill
\end{frame}

\begin{frame}{Conclusiones}
\begin{itemize}
  \item Se investigó sobre técnicas de paralelismo en GPU y CPU y su aplicabilidad en el modelo de Heisenberg.
  \item Primera paralelización utilizando \textit{straightforward parallelism} con OpenMP no cumplió las expectativas.
  \item Se ofreció un segundo método de paralelización en OpenMP, el cual busca mantener las hebras activas. Este método presentó mejores resultados pero la estrategia utilizada no pudo ser portada a GPU.
  \item Dentro de las pruebas realizadas, se alcanzó un speedup de 30 utilizando OpenMP + SIMD, al simular 21.000 espines utilizando 6 hebras.
  \item Resultados obtenidos en GPU fueron considerablemente menores de lo esperado. Esto se debe tanto a la inclusión del término dipolar, la mayor cantidad de memoria necesitada en el modelo de Heisenberg frente a otros modelos, y a la arquitecturas de las GPUs.
\end{itemize}
\end{frame}

\begin{frame}{Conclusiones}
\begin{itemize}
  \item En GPU, resultados obtenidos utilizando CUDA son levemente mejores que los obtenidos con OpenCL.
  \item En cuanto a analisis de escalabilidad, se determinó que la solución OpenMP + SIMD es escalable y se espera alcanzar cifras mayores de speedups al simular instancias con mayor número de espines.
  \item Se determinó la ventaja de utilizar números flotantes de precisión simple y se verificó que los resultados obtenidos en estas simulaciones están en conformidad con los que se obtienen al utilizar \textit{doubles}.
  \item Se plantea como trabajo a futuro implementar el método de Monte Carlo en alguna FPGA, lo cual sería lo más cercano a construir hardware específico para tal simulación.
\end{itemize}
\end{frame}

\section{}
\begin{frame}
\vfill
%\begin{center}\begin{Huge}Vielen Dank \\[10pt]
%f\"ur ihre Aufmerksamkeit.\end{Huge}\vfill
%\end{center}
\begin{center}
\begin{Huge}
Consultas
\end{Huge}
%\vfill
\end{center}
%\vfill
% centerline centra el contenido horizontalmente
\centerline{\pgfimage[height=3.5cm]{pics/questions}}
\end{frame}

\end{document}
