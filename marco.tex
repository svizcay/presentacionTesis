\begin{frame}
\vfill
\begin{center}
\begin{block}{\begin{center}\begin{Huge}Marco teórico\end{Huge}\end{center}}
\end{block}
%\vfill
\end{center}
%\vfill
\end{frame}

\begin{frame}{Marco teórico}
\begin{block}{Modelos de espines}
\begin{itemize}
  \item Momento magnético: vector compuesto del momento angular del átomo y de su momento angular intrínseco o espín.
  \item Materiales pueden ser clasificados acorde a como sus espines reaccionan a un campo magnético externo.
  \item Materiales ferromagnéticos quedan permanentemente magnetizados al ser expuestos.
\end{itemize}
\end{block}
\end{frame}

\begin{frame}{Marco teórico}
\begin{block}{Modelos de espines - Estados fundamentales}
\begin{itemize}
  \item Modificación de las estructuras ferromagnéticas, como respuesta a un campo externo, genera diferentes estados.
\end{itemize}
\centerline{\pgfimage[height=4.5cm]{pics/configuraciones}}
\end{block}
\end{frame}

\begin{frame}{Marco teórico}
\begin{block}{Modelos de espines - Histéresis (modo de reversión)}
\begin{itemize}
  \item Forma en que el sistema recorre diferentes valores de campo desde la saturación en una dirección opuesta.
\end{itemize}

\begin{columns}

  \begin{column}{.50\textwidth}
\begin{itemize}
  \item Coercitividad: intensidad necesaria para eliminar magnetización.
  \item Magnetización de remanencia.
\end{itemize}
  \end{column}
  
  \begin{column}{.50\textwidth}
	\centerline{\pgfimage[height=4.0cm]{pics/histeresis}}
  \end{column}
  
\end{columns}




\end{block}
\end{frame}

\begin{frame}{Marco teórico}
\begin{block}{Modelos de espines}
\begin{itemize}
  \item Uso de modelos de grilla de espines (\textit{lattice spin models}).
  \item Modelos son descritos a través de un Hamiltoniano (función que determina la cantidad de energía del sistema).
  \item En este trabajo se utiliza el modelo de Heisenberg.
\end{itemize}
\end{block}
\end{frame}

\begin{frame}{Marco teórico}
\begin{block}{Modelos de espines - Energías}
\begin{itemize}
  \item Energía Dipolar.
  \item Energía de Intercambio (vecinos).
  \item Energía de Zeeman (campo externo y espín).
  \item Energía de anisotropía (dirección preferencial).
\end{itemize}

\begin{equation}
\label{formula:hamiltoniano}
    H = \frac{1}{2} \sum\limits_{i\neq j} [E_{ij}^{dip} - E_{ij}^{ex}] + \sum\limits_{i} E_i^k + \sum\limits_{i} E_i^z
\end{equation}

\end{block}
\end{frame}

\begin{frame}{Marco teórico}
\begin{block}{Simulaciones de Monte Carlo}
\begin{itemize}
  \item Simulaciones de los modelos anteriores son realizadas a través del método de Monte Carlo.
  \item Técnica matemática computarizada que hace uso de números aleatorios para simular procesos estocásticos.
  \item Trayectoria estocástica en el espacio de fase.

\end{itemize}
\begin{equation}
\label{formula:estimacionmontecarlo}
    X_{MC} = \frac{1}{M}\sum\limits_{i=1}^{M} f(x_i)
\end{equation}
\end{block}
\end{frame}

\begin{frame}{Marco teórico}
\begin{block}{Markov Chain Monte Carlo}
\begin{itemize}
  \item Uso de cadenas de Markov para samplear estados con mayor probabilidad.
  \item En base a si se cumplen ciertas propiedades, la cadena converge a una distribución estacionaria.

\end{itemize}
	\centerline{\pgfimage[height=4.0cm]{pics/markov}}
\end{block}
\end{frame}

\begin{frame}{Marco teórico}
\begin{block}{Algoritmo de Metrópolis}
\begin{itemize}
  \item Es un caso particular de \textit{Markov Chain} Monte Carlo en donde se descompone la probabilidad de transición en una probabilidad de selección un estado y luego aceptar tal transación.
  \item Lo anterior permite realizar una simplificación en donde se elimina la constante de normalización (la cual no puede ser calculada debido a la infinidad de estados posibles).

\end{itemize}
	\centerline{\pgfimage[height=3.5cm]{pics/metropolis}}
\end{block}
\end{frame}

\begin{frame}{Marco teórico}
\begin{block}{Técnicas para acelerar Monte Carlo}
\begin{itemize}
  \item Métodos de clustering: Swendsen-Wang, Wolff (utilizados en sistemas con temperatura cercana a la temperatura crítica).
  \item Parallel tempering (utilizado en sistemas en criticalidad y en modelo \textit{Spin Glass}).
  \item Descomposición en tablero de ajedrez (no considera dipolar ni respeta el balance detallado).
  \item Se opta por mantener el método de actualización secuencial (\textit{single spin flip}).
  \item Se optimiza el cálculo del Hamiltoniano, calculando el delta de energía solamente en base al espín seleccionado de forma aleatoria.
\end{itemize}
\end{block}
\end{frame}